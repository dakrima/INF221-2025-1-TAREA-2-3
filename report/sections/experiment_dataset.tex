Para realizar los experimentos, se generaron casos de prueba compuestos por pares de cadenas de texto, con el principal objetivo de evaluar el rendimiento de los algoritmos de fuerza bruta y programación dinámica. Los strings fueron generados de forma aleatoria utilizando caracteres en mayúsculas del abecedario inglés.

\vspace{0.5 cm}

Los tamaños de entrada utilizados van desde los 2 hasta los 20 caracteres por cadena, lo que fue suficiente para evidenciar diferencias de rendimiento entre ambos algoritmos sin comprometer demasiado el tiempo de ejecución del enfoque de fuerza bruta, cuya complejidad impide trabajar con longitudes demasiado largas. A partir de estos rangos, se generaron 4 pares de cadenas para cada tamaño, con el fin de obtener una medición más robusta del comportamiento del tiempo de ejecución.

\vspace{0.5 cm}

Cada caso fue almacenado en el archivo inputs.txt, siguiendo el formato solicitado en el enunciado: un número inicial K que indica la cantidad de casos, seguido por K pares de líneas que contienen la longitud y el contenido de cada cadena. Los resultados correspondientes fueron escritos en el archivo outputs.txt, y las mediciones de tiempo y de longitud fueron registradas en measurements.txt.

\vspace{0.5 cm}

Cabe destacar que los casos de prueba fueron generados automáticamente mediante un script en Python, lo que garantiza la reproducibilidad de los datos y la coherencia en el análisis. No se incorporaron casos externos al formato definido por el enunciado. Esta generación de datos permite un análisis representativo del desempeño de los algoritmos.