Este trabajo aborda el problema de detectar diferencias entre dos secuencias de texto mediante la métrica de la subsecuencia común más larga (LCS). Se comparan dos paradigmas: fuerza bruta, que examina todas las subsecuencias posibles y garantiza la corrección a costa de un crecimiento exponencial, y programación dinámica, que almacena resultados intermedios y reduce la complejidad. Las implementaciones en C++ emplean la biblioteca \texttt{chrono} para medir tiempos y scripts en Python para generar casos de prueba y gráficar resultados. En un MacBook Air M2, la fuerza bruta pasa de microsegundos a más de 40 minutos cuando $n>17$, mientras que la programación dinámica mantiene tiempos del orden de $10^{-5}$ segundos en todo el rango. La evidencia valida la teoría: la fuerza bruta solo es viable en instancias sumamente pequeñas, mientras que la programación dinámica ofrece una solución práctica y escalable para aplicaciones de procesamiento de lenguaje natural, bioinformática y corrección ortográfica.