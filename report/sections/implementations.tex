La implementación de esta tarea fue desarrollada en C++ y organizada de manera separada para facilitar la modularización entre los distintos algoritmos y el manejo de archivos de entrada/salida. A continuación, se describe la estructura general del proyecto:

\begin{itemize}
    \item \texttt{code:} Carpeta principal que contiene todo el código fuente, subdividido por paradigma.
    \begin{itemize}
        \item \texttt{brute\_force:} Contiene la implementación del algoritmo de fuerza bruta. Incluye a:
        \begin{itemize}
            \item \texttt{brute\_force.cpp:} Archivo principal que lee los datos de entrada, ejecuta el algoritmo y guarda los resultados.
            \item \texttt{algorithm/sequence\_difference.cpp y algorithm/sequence\_difference.h:} Implementan la lógica del cálculo de diferencias a partir de la LCS utilizando fuerza bruta.
            \item \texttt{makefile}
        \end{itemize}
        \item \texttt{dynamic\_programming:} Contiene la implementación del algoritmo basado en programación dinámica. Tiene una estructura similar a \texttt{brute\_force}, pero con una lógica optimizada.
    \end{itemize}
    \item \texttt{data:} Carpeta donde se organizan los datos de prueba y resultados:
    \begin{itemize}
        \item \texttt{inputs.txt:} Archivo con los casos de prueba utilizados
        \item \texttt{outputs.txt:} Resultados generados por los algoritmos.
        \item \texttt{measurements.txt:} Contiene las mediciones de tiempo de ejecución y tamaño de entrada para cada ejecución de ambos algoritmos.
        \item \texttt{plots:} Carpeta donde se guardan las gráficas generadas con Python.
    \end{itemize}
    \item \texttt{scripts:} Contiene scripts auxiliares en Python que automatizan tareas como:
    \begin{itemize}
        \item \texttt{input\_generator.py:} Genera automáticamente casos de prueba en el formato que se pide en la tarea.
        \item \texttt{plot\_generator.py:} Lee las mediciones y genera gráficos del rendimiento de los algoritmos.

    \end{itemize}
\end{itemize}

\begin{mdframed}
    \begin{center}
        {\Large \url{https://github.com/dakrima/INF221-2025-1-TAREA-2-3}}
    \end{center}
\end{mdframed}