El informe confirma la premisa planteada en la introducción: al comparar cadenas de texto, elegir bien el paradigma algorítmico marca la diferencia entre una solución altamente impractica y una herramienta realmente utilizable. Los resultados muestran que el enfoque de fuerza bruta se vuelve impracticable apenas las secuencias superan un largo de 12 carácteres, al crecer su
tiempo de ejecución de microsegundos a decenas de minutos en pocos
incrementos de tamaño. Por el contrario, la programación dinámica mantiene tiempos estables, validando en la práctica su complejidad.

\vspace{0.5 cm}

Estos hallazgos responden directamente a la premisa inicial: disponer de un método que detecte diferencias entre cadenas de forma eficiente y que escale con la longitud de entrada. Además, nos muestran la relevancia de reutilizar las subsoluciones para así lograr una mejoría ante complejidades exponenciales. En trabajos e investigaciones donde la comparación de secuencias sea una parte importante, por no decir indispensable, el paradigma
dinámico resulta completamente recomendable.

\vspace{0.5 cm}

En síntesis, el trabajo demuestra que la optimización basada en
programación dinámica logra de manera robusta los objetivos de eficiencia y escalabilidad, mientras que el paradigma de fuerza bruta es útil unicamente para instancias sumamente pequeñas. Extender este estudio a técnicas que logren reducir aún más las complejidades representa una línea natural de trabajo futuro, pero no altera la conclusión central: la selección acertada del paradigma algorítmico es el factor decisivo para abordar con éxito el problema de la comparación de secuencias.