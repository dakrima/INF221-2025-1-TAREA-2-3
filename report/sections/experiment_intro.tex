Para realizar los casos de prueba de los algoritmos implementados se utilizó Hardware, entorno de software y condiciones de entrada especificas.

\paragraph{\textbf{Descripción del Hardware.}
} \leavevmode

Se utilizó un MacBook Air Apple M2 (2022) con la especificaciones siguientes \cite{apple2024macbook}:
\begin{itemize}
    \item \textbf{Procesador:} Chip M2 Apple con 4 núcleos de rendimiento @3.49 GHz y 4 núcleos de eficiencia @2.42 GHz. GPU de 8 núcleos. Neural Engine de 16 núcleos.
    \item \textbf{Memoria RAM:} 8 GB de memoria unificada LPDDR5.
    \item \textbf{Almacenamiento SSD:} 256 GB de almacenamiento SSD.
\end{itemize}

\paragraph{\textbf{Entorno de Software.}
} \leavevmode

\begin{itemize}
    \item \textbf{Sistema Operativo:} macOS Sequoia 15.0.1.
    \item \textbf{Compilador C++:} Se utiliza el compilador g++ 16.0.0, con el estándar de C++17: -std=c++17
\end{itemize}

\paragraph{\textbf{Condiciones de entrada.}} \leavevmode

Durante los experimentos se utilizaron condiciones de entrada controladas y reproducibles, permitiendo comparar de manera consistente el comportamiento de los algoritmos evaluados bajo los mismos parámetros.

\begin{itemize}
    \item \textbf{Cadenas de texto:} Para evaluar ambos algoritmos, se generaron pares de cadenas de texto de manera aleatoria, con longitudes variables desde 2 hasta 20 caracteres. Estas cadenas fueron construidas a partir de letras mayúsculas del alfabeto inglés, utilizando un script generador programado en Python. Cada par de cadenas fue procesado siguiendo el mismo formato especificado en el enunciado de la tarea: para cada caso, se almacena la longitud de la cadena, seguida de la cadena misma en una línea separada.
\end{itemize}