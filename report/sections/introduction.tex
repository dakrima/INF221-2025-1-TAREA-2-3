El presente trabajo se situa en el campo del Análisis y Diseño de Algoritmos en Ciencias de la Computación, un área fundamental para resolver problemas computacionales de manera eficaz y eficiente. Dentro de este campo, los problemas de edición y comparación de cadenas de texto han recibido gran atención, especialmente en aplicaciones que requieren medir similitudes entre secuencias, como el procesamiento de lenguaje natural, la bioinformática \cite{contreras2018algoritmos} y la corrección ortográfica. Entre las distintas métricas para medir similitud o diferencia entre secuencias, el análisis de la subsecuencia común más larga (LCS, por sus siglas en inglés) destaca como una herramienta fundamental.

\vspace{0.5 cm}

Una de las estrategias más utilizadas para comparar cadenas es el uso de la subsecuencia común más larga, la cual permite identificar los segmentos que se comparten entre dos secuencias. A partir de la LCS, es posible deducir los fragmentos que difieren entre ambas, lo que permite no solo cuantificar su diferencia sino también obtener una representación visual de estas mismas.

\vspace{0.5 cm}

El propósito de este informe es estudiar y comparar dos enfoques algorítmicos para resolver este problema: el método de fuerza bruta, basado en una búsqueda exhaustiva de todas las subsecuencias comunes, y el método de programación dinámica, que optimiza la búsqueda al ir recordando lo que ya ha calculado, en lugar de volver a hacerlo desde cero. Ambos algoritmos serán evaluados en cuanto a su eficiencia y efectividad para detectar las diferencias entre dos cadenas de texto, analizando cómo escalan en tiempo frente a entradas de diferentes tamaños.

\vspace{0.5 cm}

Este estudio permite introducir conceptos avanzados como el uso de programación dinámica para optimizar procesos combinatorios, y explorar las ventajas y limitaciones de los enfoques de fuerza bruta frente a técnicas algorítmicas más eficientes. El informe tiene como objetivo proporcionar una visión tanto práctica como teórica de cada paradigma, ofreciendo una base sólida para el análisis de algoritmos en contextos donde la comparación entre secuencias es una operación fundamental.
